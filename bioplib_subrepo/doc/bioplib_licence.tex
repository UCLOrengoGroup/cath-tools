\documentclass[12pt]{article}
\usepackage{a4}
\parskip 1em

% Put the licensee's name and address and the date here:
% ------------------------------------------------------
\newcommand{\licensee}{\rule{5.3in}{0.5pt}\mbox{}}
\renewcommand{\date}{\rule{1.5in}{0.5pt}\mbox{}}







\begin{document}
% \thispagestyle{empty}
% 
% {\sffamily\Large
% \noindent To: \licensee
% 
% \noindent From: Dr.\ Andrew C.\ R.\ Martin, U.C.L.
% 
% \noindent Fax: +44 (0) 1372 813069
% \vspace{1in}
% 
% 
% The agreement for using my Bioplib software follows. Please return by
% post to the address show on the agreement. To receive the software as
% soon as possible, please also FAX to the number shown above.
% }
% 
% 
% \newpage
% \setcounter{page}{1}

\begin{center}
{\large\bfseries Bioplib}

\copyright 1990--2006 Dr.\ Andrew C.\ R.\ Martin

23 Stag Leys, Ashtead, Surrey, KT21 2TD, UK.

EMail: andrew@bioinf.org.uk

\end{center}
\vspace{1em}

This agreement is made on \date\ between:
\begin{itemize}
\item Dr.\ Andrew C.\ R.\ Martin (hereinafter called the LICENSOR) and
\item \licensee\ (hereinafter called the LICENSEE)
\end{itemize}

Bioplib is a library of routines for the manipulation of protein
structure and sequence using the C programming language. 

In this agreement, the term `Bioplib' refers to link libraries, and to
the source code and object code which comprises these
libraries. Should the organisation of the code into libraries change
in the future, the term `Bioplib' will refer to any such link
libraries (and associated source and object code) which comprise the
future release of a package described by the author as `Bioplib'.
The current organisation of the software is described in Appendix~I.

The term `compiling' is used to describe the process of compiling and
linking software on a computer to form an executable program.

Bioplib was written by and is copyright \copyright 1990-2006, Dr.\
Andrew C.\ R.\ Martin. The bulk of Bioplib was written by Dr.\ Martin
while self-employed although numerous additions and enhancements have
been made while Dr.\ Martin was an employee of either University
College London or The University of Reading. Under the terms of the
original Bioplib licence, all copyright and other intellectual
property rights remain with the author. Bioplib may not be used for
any purposes other than as described in the following terms and
conditions:--

\begin{enumerate}
\item Bioplib may only be used by the LICENSEE and members of the
LICENSEE's laboratory for compiling software provided by the
LICENSOR and for compiling software developed in the LICENSEE's
laboratory for research carried out there.
Bioplib will be used only by those members of the LICENSEE's laboratory
to whom it must reasonably be communicated to enable research to
be undertaken and who agree to be bound by the same conditions.
The LICENSEE shall procure and enforce such agreement from members of
the LICENSEE's laboratory for the benefit of the LICENSOR.

\item The publication of research using software depending on Bioplib 
must reference the Bioplib library and the LICENSOR. This agreement may 
be taken as permission for citation as a `Personal Communication' from 
the LICENSOR.

\item All software compiled with Bioplib must include reference to Bioplib 
and the LICENSOR in any copyright messages.

\item All forms of Bioplib will be kept in a reasonably secure place to
prevent unauthorised access.

\item Bioplib may not be copied for distribution to any third
party except as expressly described below.

\item The complete Bioplib library may be distributed to third
parties, \emph{unmodified}, solely for the purpose of compiling
software developed by the LICENSEE. The LICENSEE must make clear to
the third party in the documentation accompanying the LICENSEE's
software, that the Bioplib library has been supplied by the LICENSOR
and that the third party may only use Bioplib for the purposes of
compiling the LICENSEE's software. The third party may not
redistribute Bioplib in any circumstances except as part of the
LICENSEE's software. Third parties wishing to use Bioplib as part of
their own software should contact the LICENSOR.  A file describing
these conditions to third parties is included with Bioplib.

\item Any modifications or changes made to Bioplib should be sent to
the LICENSOR for possible inclusion in future versions of Bioplib. Any
such changes which are incorporated into Bioplib will be
acknowledged, but become the property of the LICENSOR.

\item Bioplib shall be used exclusively for academic teaching and
research. Bioplib will not be used for any commercial research or
research associated with an industrial company unless a separate 
agreement is made with the LICENSOR.

\item Bioplib may not be used as part of any software which is sold
for more than a nominal sum to cover copying and distribution
costs. Should the LICENSEE wish to sell for profit software which
relies on Bioplib, or any part thereof, then a separate agreement must
be made with the LICENSOR.

\end{enumerate}
\newpage

\noindent{\large\bfseries Appendix I}

\noindent At present (September, 2006), Bioplib is split into two link libraries:
\begin{description}
\item[libbiop.a] A library of routines specifically for the
manipulation of protein structures,
\item[libgen.a] A library of general purpose C routines.
\end{description}

\vspace{0.5in}

\noindent{\large\bfseries Signed:}

\noindent I, (\licensee), the LICENSEE agree to the terms herein.

\noindent For and on behalf of (name and department address):\\
\vskip 0.25in
\noindent\rule{5in}{0.5pt}
\vskip 0.25in
\noindent\rule{5in}{0.5pt}
\vskip 0.25in
\noindent\rule{5in}{0.5pt}
\vskip 0.25in
\noindent\rule{5in}{0.5pt}
\vskip 0.25in

\noindent Signed: \rule{3in}{0.5pt}
\vskip 0.25in

\noindent Dated: \rule{3in}{0.5pt}
\vskip 0.25in

\noindent EMail address: \rule{3in}{0.5pt}


\end{document}

